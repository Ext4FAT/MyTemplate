%  -*- coding: utf-8 -*-
% !TEX program = xelatex

\documentclass[xcolor=table,notheorems,compress,blue]{beamer}
\usetheme{Boadilla}
%\usetheme[color=blue]{epyt} % black, blue, green, white
\usepackage[UTF8,noindent]{ctex}

\useoutertheme[subsection=false]{smoothbars}
\setbeamertemplate{footline}{}
\setbeamercolor{background canvas}{bg=}



% include packages
\usepackage{float} 
\usepackage{algorithmic}
\usepackage{algorithm}
\usepackage{subfigure}
\usepackage{multicol}
\usepackage{amsmath}
\usepackage{graphicx}
\usepackage{epstopdf}
\usepackage{wallpaper}
\usepackage{url}
%\usepackage{multimedia}
\usepackage{hyperref}
\usepackage{colortbl}
\usepackage{fancyhdr}
\usepackage{eso-pic}
%%% store graphics in a box, background
\newsavebox{\mygraphic}
\sbox{\mygraphic}{%
	\includegraphics[keepaspectratio,height=\textheight,width=\textwidth]{./Picture/Material/MMM2017.png}}
\AddToShipoutPicture{
	\put(20,40){\usebox{\mygraphic}}
	%put(-2.2,6){\makebox(-2.2,-6)[bl]{\usebox{\mygraphic}}}
}


%\hypersetup{
%  pdfpagemode={FullScreen},
%}

\newtheorem{theorem}{定理}
\newtheorem{definition}[theorem]{定义}
\newtheorem{example}[theorem]{例子}

\newtheorem*{theorem*}{定理}
\newtheorem*{definition*}{定义}
\newtheorem*{example*}{例子}

\renewcommand{\proofname}{证明}

\title{\textbf{\Huge{基于深度相机的场景物体定位与抓取研究}}}
\author{\textbf{\huge{周逸徉}}}
%\author{\textbf{\huge{导师: 路通 教授}}}

\begin{document}

\begin{frame}[plain]\transboxout
\titlepage
\end{frame}

\begin{frame}\transboxin
\begin{center}
\tableofcontents[hideallsubsections]
\end{center}
\end{frame}

\section{引子}
	\begin{frame}{课题引入}\transdissolve
		\begin{exampleblock}{机器人领域蓬勃发展}
			
		\end{exampleblock}
		\begin{exampleblock}{视觉伺服兴起}
			
		\end{exampleblock}
		\begin{exampleblock}{机械臂抓取任务如火如荼}
			
		\end{exampleblock}
%		机器人领域的蓬勃发展\pause
%		\begin{itemize}[<+->]
%			\item 结构简洁,只有包含必需元素的底栏,没有顶栏和侧栏。
%			\item 内容简洁,列表环境和定理环境都使用了简单的形式。
%			\item 配色简洁,仅仅用到几种背景色和前景色。
%		\end{itemize}
	\end{frame}

\section{基于二维的物体候选区域筛选}
  \subsection{基于深度信息的区域分割}
	\begin{frame}{区域分割}
		\begin{exampleblock}{流程展示}
			\begin{figure}[htpb]
				\centering
				\begin{minipage}[b]{0.8in}
					\centerline{ \includegraphics[width=0.8in, height=.65in]{./Picture/Example/depth.eps} }
				\end{minipage}
				\begin{minipage}[b]{0.8in}
					\centerline{ \includegraphics[width=0.6in, height=.35in]{./Picture/Material/arrow.jpg} }
					\centerline{\tiny{区域增长}}
				\end{minipage}
				\begin{minipage}[b]{0.8in}
					\centerline{ \includegraphics[width=0.8in, height=.65in]{./Picture/Example/merge.eps} }
				\end{minipage}
				\begin{minipage}[b]{0.8in}
					\centerline{ \includegraphics[width=0.6in, height=.35in]{./Picture/Material/arrow.jpg} }
					\centerline{\tiny{区域合并}}
				\end{minipage}
				\begin{minipage}[b]{0.8in}
					\centerline{ \includegraphics[width=0.8in, height=.65in]{./Picture/Example/seg.eps} }
				\end{minipage}
			\end{figure}
		\end{exampleblock}
		\begin{exampleblock}{输入}
			\begin{itemize}		
				\item  深度信息: $Depth$
				\item  主分割数目: ${topk}$
			\end{itemize}
		\end{exampleblock}
		\begin{exampleblock}{输出}
			\begin{itemize}		
				\item  主分割集: ${MS}$
			\end{itemize}
		\end{exampleblock}
	\end{frame}
	
	\begin{frame}{区域分割}
		\vspace{-10pt}
		\begin{exampleblock}{过程}
		\end{exampleblock}
		\vspace{-18pt}
		\begin{exampleblock}{\small{PART 1: 粗略分割}}
			\begin{itemize}
				\item 在$Depth$上使用区域增长算法找出全黑分割集$BS$和深度分割集$DS$ 
				\item 将$DS$中的点集按点的数目降序排列
				\item 从$DS$中选出前$topk$个分割点集作为主分割集$MS$
			\end{itemize}
		\end{exampleblock}
		\vspace{-15pt}
		\begin{exampleblock}{\small{PART 2: 区域合并}}
			\textbf{for} {$seg \in  (DCR - MS) \cup BCR$}
			\\ \quad \textbf{for} {$p \in seg$}
			%		\begin{itemize}
			\\ \quad \quad $CS \leftarrow \{ms|ms \in MS \wedge p\ in\ ConvexHull(ms)\}$
			\\ \quad \quad $SEG \leftarrow \mathop{\arg\min}_{ms}(distance(ms), ms\in CS)$	
			\\ \quad \quad $SEG \leftarrow SEG \cup \{p\}$
			%		\end{itemize}
			\\ \quad \textbf{end for}
			\\ \textbf{end for} 
			\\ \textbf{return} $MS$
		\end{exampleblock}
	\end{frame}
	
  \subsection{基于颜色数据的物体分类}
	\begin{frame}{物体分类}
		\begin{exampleblock}{流程展示}
			\begin{figure}[htpb]
				\centering
				\begin{minipage}[b]{0.8in}
					\centerline{ \includegraphics[width=0.8in, height=.65in]{./Picture/Example/color.eps} }
				\end{minipage}
				\begin{minipage}[b]{0.8in}
					\centerline{ \includegraphics[width=0.6in, height=.35in]{./Picture/Material/arrow.jpg} }
					\centerline{\tiny{矩形绑定框}}
				\end{minipage}
				\begin{minipage}[b]{0.8in}
					\centerline{ \includegraphics[width=0.8in, height=.65in]{./Picture/Example/regions.eps} }
				\end{minipage}
				\begin{minipage}[b]{0.8in}
					\centerline{ \includegraphics[width=0.6in, height=.35in]{./Picture/Material/arrow.jpg} }
					\centerline{\tiny{分类器筛选}}
				\end{minipage}
				\begin{minipage}[b]{0.8in}
					\centerline{ \includegraphics[width=0.8in, height=.65in]{./Picture/Example/classification.eps} }
				\end{minipage}
			\end{figure}
		\end{exampleblock}
		\begin{exampleblock}{输入}
			\begin{itemize}		
				\item 颜色数据: $Color$	
				\item 主分割集: $MS$ 
				\item 物体分类器: ${Classifier}$
			\end{itemize}
		\end{exampleblock}
		\begin{exampleblock}{输出}
			\begin{itemize}
				\item 候选区域集: $Candidate$
			\end{itemize}
		\end{exampleblock}
	\end{frame}
	
	\begin{frame}{物体分类}
		\vspace{-7pt}
		\begin{exampleblock}{过程}
			$Rect \leftarrow \{r|r=BoundBox(ms), ms\in MS\}$
			\\
			{$Candidate \leftarrow \emptyset$}
			\\ \textbf{for} {$rect \in Rect$}
			\\ \quad {$region \leftarrow$ Extract-Region$(Color, rect)$}
			%\\ \quad {${Region}_{64\times64} \leftarrow$ Resize$({Region}_{color})$}
			\\ \quad {${feature} \leftarrow$ Calculate-HOG$(region)$}
			\\ \quad {$label \leftarrow$ Classify$({Classifier}, {feature})$}
			\\ \quad \textbf{if}  {$label \neq 0$}
			\\ \quad \quad {$Candidate \leftarrow \{<rect,label>\} \cup Candidate$}
			\\ \quad \textbf{end if}
			\\ \textbf{end for}
			\\ \textbf{return} $Candidate$
		\end{exampleblock}
	\end{frame}
	
	
\section{基于三维的物体点云配准}
  \subsection{点云配准}
	\begin{frame}{点云配准}
		\begin{exampleblock}{流程展示}
			\begin{figure}[htpb]
				\centering
				\begin{minipage}[b]{0.8in}
					\centerline{ \includegraphics[width=0.8in, height=.65in]{./Picture/Material/realsense.jpg} }
				\end{minipage}
				\begin{minipage}[b]{0.8in}
					\centerline{ \quad}
					\centerline{ \includegraphics[width=0.6in, height=.35in]{./Picture/Material/arrow.jpg} }
					\centerline{\tiny{点云生成}}
				\end{minipage}
				\begin{minipage}[b]{0.8in}
					\centerline{ \includegraphics[width=0.8in, height=.65in]{./Picture/Example/pointcloud.eps} }
				\end{minipage}
				\begin{minipage}[b]{0.8in}
					\centerline{ \includegraphics[width=0.6in, height=.35in]{./Picture/Material/arrow.jpg} }
					\centerline{\tiny{加载模型}}
				\end{minipage}
				\begin{minipage}[b]{0.8in}
					\centerline{ \includegraphics[width=0.8in, height=.65in]{./Picture/Example/before-match.png} }
				\end{minipage}
			\end{figure}
			\vspace{-25pt}
			\begin{figure}[htpb]
				\centering
				\begin{minipage}[b]{0.8in}
					\centerline{ \includegraphics[width=0.8in, height=.65in]{./Picture/Material/transparent.png} }
				\end{minipage}
				\begin{minipage}[b]{0.8in}
					\centerline{ \includegraphics[width=0.8in, height=.65in]{./Picture/Material/transparent.png} }
				\end{minipage}
				\begin{minipage}[b]{0.8in}
					\centerline{ \includegraphics[width=0.8in, height=.65in]{./Picture/Material/transparent.png} }
				\end{minipage}
				\begin{minipage}[b]{0.8in}
					\centerline{ \includegraphics[width=0.8in, height=.65in]{./Picture/Material/transparent.png} }
				\end{minipage}
				\begin{minipage}[b]{0.6in}
					\centerline{ \includegraphics[width=.45in, height=.6in]{./Picture/Material/down.jpg} }
				\end{minipage}
			\end{figure}
			\vspace{-50pt} 
			\hspace{-65pt}
			\rightline{\tiny{RANSAC\quad}}
			\vspace{15pt} 
			\begin{figure}[htpb]
				\centering
				\begin{minipage}[b]{0.8in}
%					\textbf{\huge{
%					\begin{equation}
%						\begin{bmatrix} R & t\\ 0 & 1 \end{bmatrix}
%					\end{equation}
%					}
%					}
					%\label{PROBLEM-------cann't show matirx img}
					\centerline{ \includegraphics[width=0.65in, height=.65in]{./Picture/Material/matrix.png} }
				\end{minipage}
				\begin{minipage}[b]{0.8in}
					\centerline{ \quad}
					\centerline{ \includegraphics[width=0.6in, height=.35in]{./Picture/Material/left.jpg} }
					\centerline{\tiny{变换矩阵}}
				\end{minipage}
				\begin{minipage}[b]{0.8in}
					\centerline{ \includegraphics[width=0.8in, height=.65in]{./Picture/Example/match.png} }
				\end{minipage}
				\begin{minipage}[b]{0.8in}
					\centerline{ \includegraphics[width=0.6in, height=.35in]{./Picture/Material/left.jpg} }
					\centerline{\tiny{ICP}}
				\end{minipage}
				\begin{minipage}[b]{0.8in}
					\centerline{ \includegraphics[width=0.8in, height=.65in]{./Picture/Example/mid.png} }
				\end{minipage}
			\end{figure}
		\end{exampleblock}
	\end{frame}
	\begin{frame}{点云配准}
		\begin{exampleblock}{Input}
			\begin{itemize}		
				\item  区域点云数据: $Points Cloud$
				\item  3D物体点云数据: $Model_{label}$
				\item  最小相似度: $sim$
				\item  最大可接受距离: $mad$
				\item  最大迭代次数: $mi$ 
				\item  接受比例: $an$
			\end{itemize}
		\end{exampleblock}
		\begin{exampleblock}{Output}
			\begin{itemize}
				\item Rotate Matrix: $R$
				\item Translation Vector: $\vec t$ 
			\end{itemize}
		\end{exampleblock}
	\end{frame}
	
	\begin{frame}{Registration}
		\vspace{-9pt}
		\begin{exampleblock}{过程}
		\end{exampleblock}
		\vspace{-18pt}
		\begin{exampleblock}{\small{PART 1: RANSAC}}
			$R\leftarrow dig(1,1,1)$, $\vec t\leftarrow (0,0,0)^T$ 
			\\ \textbf{repeat} 
			\begin{itemize}
				\item $mrs \leftarrow $ Random-Select$(Model_{label})$
				\item $pcrs \leftarrow$ Random-Select$(Points Cloud)$
				\item $similar$-$pairs \leftarrow \{<p1,p2>|Similarity(FPFH_{p1},FPFH_{p2})>sim  \wedge p1 \in mrs \wedge p2 \in pcrs\}$
				\item $R$, $\vec t$ $\leftarrow$ Estimate-TransformMatrix$(similar$-$pairs)$
				\item $inliers \leftarrow \{p| <p,*> \in $ $similar$-$pairs\}$
				\item $alsoinliers \leftarrow \{p| min (distance(R*p+\vec t,q), q\in pcrs) < mad \wedge p \in mrs-inliers \}$
			\end{itemize}
			\textbf{until} \quad $|alsoinliers| > an$ or Reach $mi$
		\end{exampleblock}
	\end{frame}
	\begin{frame}{Registration}
		\begin{exampleblock}{过程 (\emph{续} )}
		\end{exampleblock}
		\vspace{-20pt}
		\begin{exampleblock}{\small{PART 2: ICP}}
			$R_{icp}$, $\vec t_{icp} \leftarrow Classic$-$ICP(R * Model_{label}, Points Cloud)$
			\\ $R \leftarrow R_{icp} * R$
			\\ $\vec t \leftarrow R_{icp} * \vec t + \vec t_{icp}$
			\\ \textbf{return} $R$, $\vec t$
		\end{exampleblock}
	\end{frame}	



\section{物体抓取系统的实现}	
	\begin{frame}{算法流程}
		
	\end{frame}	
	\begin{frame}{坐标转换}
		
	\end{frame}	
	\begin{frame}{抓取演示}
		
	\end{frame}	


\section{总结与展望}	
\begin{frame}{总结}
	
\end{frame}	
\begin{frame}{展望}
	
\end{frame}	


%%%%%%%%%%%%%%%%%%%%%%%%%%%%%%%%%%%%%%%%%%%%%%%%%%%%%%%%%%%%%%%%%%%%%%%%%
\section{}
%%%%%%%%%%%%%%%%%%%%%%%%%%%%%%%%%%%%%%%%%%%%%%%%%%%%%%%%%%%%%%%%%%%%%%%%%
\begin{frame}[plain]
	\thispagestyle{empty}
	\begin{columns}
		\begin{column}{\paperwidth}
			\includegraphics[width=\paperwidth,height=\paperheight]{./Picture/Material/3Q.png}
		\end{column}
	\end{columns}
\end{frame}
\begin{frame}[plain]
	\thispagestyle{empty}
	\begin{columns}
		\begin{column}{\paperwidth}
			\includegraphics[width=\paperwidth,height=\paperheight]{./Picture/Material/QA.jpg}
		\end{column}
	\end{columns}
\end{frame}
%%%%%%%%%%%%%%%%%%%%%%%%%%%%%%%%%%%%%%%%%%%%%%%%%%%%%%%%%%%%%%%%%%%%%%%%%
\end{document}