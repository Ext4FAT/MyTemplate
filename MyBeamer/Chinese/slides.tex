%  -*- coding: utf-8 -*-
% !TEX program = xelatex

\documentclass[xcolor=table,compress,blue]{beamer}
\mode<presentation>
\hypersetup{
  pdfpagemode={FullScreen},
}



\usetheme{Boadilla}
% define your own colours:
\definecolor{Red}{rgb}{1,0,0}
\definecolor{Blue}{rgb}{0,0,1}
\definecolor{Green}{rgb}{0,1,0}
\definecolor{magenta}{rgb}{1,0,.6}
\definecolor{lightblue}{rgb}{0,.5,1}
\definecolor{lightpurple}{rgb}{.6,.4,1}
\definecolor{gold}{rgb}{.6,.5,0}
\definecolor{orange}{rgb}{1,0.4,0}
\definecolor{hotpink}{rgb}{1,0,0.5}
\definecolor{newcolor2}{rgb}{.5,.3,.5}
\definecolor{newcolor}{rgb}{0,.3,1}
\definecolor{newcolor3}{rgb}{1,0,.35}
\definecolor{darkgreen1}{rgb}{0, .35, 0}
\definecolor{darkgreen}{rgb}{0, .6, 0}
\definecolor{darkred}{rgb}{.75,0,0}

\definecolor{olive}{cmyk}{0.64,0,0.95,0.4}
\definecolor{purpleish}{cmyk}{0.75,0.75,0,0}

%\usetheme[color=blue]{epyt} % black, blue, green, white
\usepackage[UTF8,noindent]{ctex}

\useoutertheme[subsection=false]{smoothbars}
\setbeamertemplate{footline}{}
\setbeamercolor{background canvas}{bg=}



% include packages
\usepackage{float} 
\usepackage{algorithmic}
\usepackage{algorithm}
\usepackage{subfigure}
\usepackage{multicol}
\usepackage{amsmath,bm}
\usepackage{graphicx}
\usepackage{epstopdf}
\usepackage{wallpaper}
\usepackage{url}
\usepackage{multimedia}
%\usepackage{movie15}
\usepackage{hyperref}
\usepackage{colortbl}
\usepackage{fancyhdr}
\usepackage{eso-pic}
\usepackage{ctable}
\usepackage{multirow}

\usepackage[english]{babel}
%\usepackage[latin1]{inputenc}
%\usepackage{times}
\usepackage[T1]{fontenc}
\usepackage{animate}


\usepackage{tikz}
\usetikzlibrary{arrows, shapes}
\tikzstyle{block}=[draw opacity=0.7,line width=1.4cm]

% my custom config
\usepackage{amsmath}
\DeclareMathOperator*{\argmax}{\argmax}
\setcounter{tocdepth}{3}
\renewcommand{\algorithmicrequire}{\textbf{Input:}}
\renewcommand{\algorithmicensure}{\textbf{Output:}}


%%% store graphics in a box, background
\newsavebox{\mygraphic}
\sbox{\mygraphic}{%
	\includegraphics[width=\textwidth, height=\textheight]{./Picture/Material/NJU.png}}
\AddToShipoutPicture{
	\put(0.,3.){\usebox{\mygraphic}}
	%put(-2.2,6){\makebox(-2.2,-6)[bl]{\usebox{\mygraphic}}}
}


%\hypersetup{
%  pdfpagemode={FullScreen},
%}

\AtBeginSection[] {
	\begin{frame}[plain]\transboxout
		\frametitle{提纲}
		\vspace{-16pt}
		\textbf{\Large{\tableofcontents[currentsection,hideallsubsections]}}
	\end{frame}
	\addtocounter{framenumber}{-1}
}

%\titlegraphic{\centering \includegraphics[height=1.5cm]{./Picture/Logo/nju.jpg}}


\newcommand{\spacetwo}{\textcolor{white}{\\sp\\ace}}
\newcommand{\spaceone}{\textcolor{white}{sp\\ace}}
\title{\textbf{\Huge{基于深度相机的场景物体定位与抓取研究}}}
\institute{ 
	\textbf{\large{Social University}}
}
\author{\textbf{\huge{IDLER}}}
\date{\tiny{}}

\begin{document}
\frame{
	\titlepage
	\thispagestyle{empty}
}
\begin{frame}[plain]{提纲}
	\vspace{-16pt}
	\textbf{\Large{\tableofcontents[hideallsubsections]}}
\end{frame}


%\begin{frame}[plain]\transboxout
%	\titlepage
%\end{frame}
%
%\begin{frame}\transboxin
%	\begin{center}
%		\tableofcontents[hideallsubsections]
%	\end{center}
%\end{frame}

\section{课题背景}
  \subsection{绪论}
	\begin{frame}{课题引入}
		\begin{exampleblock}{机械臂自动抓取客观困难}
			利用相机进行机械臂自动抓取驱动是一个很难的课题,纯RGB相机在解决此任务时会遇到诸多困难。在确定物体大小、距离远近以及方向时,尤为突出。复
			杂算法以适应实际场景则时间复杂度陡增。
		\end{exampleblock}
		\begin{exampleblock}{深度相机的广泛应用}
			在机器人研究领域,越来越多的应用开始RGB-D的深度相机,例如室内场景3D建模,SLAM,机器人自动定位与导航等。
		\end{exampleblock}
		\begin{exampleblock}{廉价的可行方案}
			由于抓取任务客观存在诸多困难,同时RGB-D的深度相机又广泛的使用,能不能提出一个廉价可用的机械臂抓取方案?
		\end{exampleblock}
	\end{frame}
	\subsection{提出的算法流程}
		\begin{frame}{提出的算法框架}
		\centering{
			\includegraphics[width=\textwidth ]{./Picture/Example/flow.eps}
		}
		\end{frame}	
  
\section{结合深度和颜色信息的场景物体检测}
  \subsection{基于深度信息的区域分割}
	\begin{frame}{区域分割}
		\begin{exampleblock}{流程展示}
			\begin{figure}[htpb]
				\centering
				\begin{minipage}[b]{0.8in}
					\centerline{ \includegraphics[width=0.8in, height=.65in]{./Picture/Example/depth.eps} }
				\end{minipage}
				\begin{minipage}[b]{0.8in}
					\centerline{ \includegraphics[width=0.6in, height=.35in]{./Picture/Material/arrow.jpg} }
					\centerline{\tiny{区域增长}}
				\end{minipage}
				\begin{minipage}[b]{0.8in}
					\centerline{ \includegraphics[width=0.8in, height=.65in]{./Picture/Example/merge.eps} }
				\end{minipage}
				\begin{minipage}[b]{0.8in}
					\centerline{ \includegraphics[width=0.6in, height=.35in]{./Picture/Material/arrow.jpg} }
					\centerline{\tiny{区域合并}}
				\end{minipage}
				\begin{minipage}[b]{0.8in}
					\centerline{ \includegraphics[width=0.8in, height=.65in]{./Picture/Example/seg.eps} }
				\end{minipage}
			\end{figure}
		\end{exampleblock}
		\begin{exampleblock}{输入}
			\begin{itemize}		
				\item  深度信息: $Depth$
				\item  主分割数目: ${topk}$
			\end{itemize}
		\end{exampleblock}
		\begin{exampleblock}{输出}
			\begin{itemize}		
				\item  主分割集: ${MS}$
			\end{itemize}
		\end{exampleblock}
	\end{frame}
	
	\begin{frame}{区域分割}
		\vspace{-10pt}
		\begin{exampleblock}{过程}
		\end{exampleblock}
		\vspace{-18pt}
		\begin{exampleblock}{\small{PART 1: 粗略分割}}
			\begin{itemize}
				\item 在$Depth$上使用区域增长算法找出全黑分割集$BS$和深度分割集$DS$ 
				\item 将$DS$中的点集按点的数目降序排列
				\item 从$DS$中选出前$topk$个分割点集作为主分割集$MS$
			\end{itemize}
		\end{exampleblock}
		\vspace{-15pt}
		\begin{exampleblock}{\small{PART 2: 区域合并}}
			\textbf{for} {$seg \in  (DCR - MS) \cup BCR$}
			\\ \quad \textbf{for} {$p \in seg$}
			%		\begin{itemize}
			\\ \quad \quad $CS \leftarrow \{ms|ms \in MS \wedge p\ in\ ConvexHull(ms)\}$
			\\ \quad \quad $SEG \leftarrow \mathop{\arg\min}_{ms}(distance(ms), ms\in CS)$	
			\\ \quad \quad $SEG \leftarrow SEG \cup \{p\}$
			%		\end{itemize}
			\\ \quad \textbf{end for}
			\\ \textbf{end for} 
			\\ \textbf{return} $MS$
		\end{exampleblock}
	\end{frame}
	
  \subsection{基于颜色数据的物体分类}
	\begin{frame}{物体分类}
		\begin{exampleblock}{流程展示}
			\begin{figure}[htpb]
				\centering
				\begin{minipage}[b]{0.8in}
					\centerline{ \includegraphics[width=0.8in, height=.65in]{./Picture/Example/color.eps} }
				\end{minipage}
				\begin{minipage}[b]{0.8in}
					\centerline{ \includegraphics[width=0.6in, height=.35in]{./Picture/Material/arrow.jpg} }
					\centerline{\tiny{矩形绑定框}}
				\end{minipage}
				\begin{minipage}[b]{0.8in}
					\centerline{ \includegraphics[width=0.8in, height=.65in]{./Picture/Example/regions.eps} }
				\end{minipage}
				\begin{minipage}[b]{0.8in}
					\centerline{ \includegraphics[width=0.6in, height=.35in]{./Picture/Material/arrow.jpg} }
					\centerline{\tiny{分类器筛选}}
				\end{minipage}
				\begin{minipage}[b]{0.8in}
					\centerline{ \includegraphics[width=0.8in, height=.65in]{./Picture/Example/classification.eps} }
				\end{minipage}
			\end{figure}
		\end{exampleblock}
		\begin{exampleblock}{输入}
			\begin{itemize}		
				\item 颜色数据: $Color$	
				\item 主分割集: $MS$ 
				\item 物体分类器: ${Classifier}$
			\end{itemize}
		\end{exampleblock}
		\begin{exampleblock}{输出}
			\begin{itemize}
				\item 候选区域集: $Candidate$
			\end{itemize}
		\end{exampleblock}
	\end{frame}
	
	\begin{frame}{物体分类}
		\vspace{-7pt}
		\begin{exampleblock}{过程}
			$Rect \leftarrow \{r|r=BoundBox(ms), ms\in MS\}$
			\\
			{$Candidate \leftarrow \emptyset$}
			\\ \textbf{for} {$rect \in Rect$}
			\\ \quad {$region \leftarrow$ Extract-Region$(Color, rect)$}
			%\\ \quad {${Region}_{64\times64} \leftarrow$ Resize$({Region}_{color})$}
			\\ \quad {${feature} \leftarrow$ Calculate-HOG$(region)$}
			\\ \quad {$label \leftarrow$ Classify$({Classifier}, {feature})$}
			\\ \quad \textbf{if}  {$label \neq 0$}
			\\ \quad \quad {$Candidate \leftarrow \{<rect,label>\} \cup Candidate$}
			\\ \quad \textbf{end if}
			\\ \textbf{end for}
			\\ \textbf{return} $Candidate$
		\end{exampleblock}
	\end{frame}
	\subsection{实验结果}
	\begin{frame}{实验结果}
		\begin{exampleblock}{不同尺度提取HOG的分类结果}
			\centering{
				\includegraphics[width=\textwidth]{./Picture/Result/classification.png}
			}
		\end{exampleblock}
	\end{frame}
	
\section{结合RANSAC与ICP的场景物体点云配准}
  \subsection{快速点特征直方图}
  	\begin{frame}{快速点特征直方图(FPFH)}
  		\begin{exampleblock}{FPFH特征描述子}
  			\begin{figure}[htpb]
	  			\centering
  			%	\includegraphics[width=0.45\textwidth]{./Picture/FPFH/pfh_diagram.png}
  				\includegraphics[width=0.45\textwidth]{./Picture/FPFH/pfh_frame.png}
  				\includegraphics[width=0.35\textwidth]{./Picture/FPFH/fpfh_diagram.png}
  			\end{figure}
  		\end{exampleblock}
  		\begin{exampleblock}{统计FPFH中d维分离直方图}
  			\begin{figure}[htpb]
			 	\centering
			 	\includegraphics[width=0.5\textwidth]{./Picture/FPFH/fpfh_theory.jpg}
		 	\end{figure}
  		\end{exampleblock}
  	
  	
	\end{frame}
  \subsection{结合RANSAC+ICP的点云配准}
	\begin{frame}{点云配准}
		\begin{exampleblock}{流程展示}
			\begin{figure}[htpb]
				\centering
				\begin{minipage}[b]{0.8in}
					\centerline{ \includegraphics[width=0.8in, height=.65in]{./Picture/Material/realsense.jpg} }
				\end{minipage}
				\begin{minipage}[b]{0.8in}
					\centerline{ \quad}
					\centerline{ \includegraphics[width=0.6in, height=.35in]{./Picture/Material/arrow.jpg} }
					\centerline{\tiny{点云生成}}
				\end{minipage}
				\begin{minipage}[b]{0.8in}
					\centerline{ \includegraphics[width=0.8in, height=.65in]{./Picture/Example/pointcloud.eps} }
				\end{minipage}
				\begin{minipage}[b]{0.8in}
					\centerline{ \includegraphics[width=0.6in, height=.35in]{./Picture/Material/arrow.jpg} }
					\centerline{\tiny{加载模型}}
				\end{minipage}
				\begin{minipage}[b]{0.8in}
					\centerline{ \includegraphics[width=0.8in, height=.65in]{./Picture/Example/before-match.png} }
				\end{minipage}
			\end{figure}
			\vspace{-25pt}
			\begin{figure}[htpb]
				\centering
				\begin{minipage}[b]{0.8in}
					\centerline{ \includegraphics[width=0.8in, height=.65in]{./Picture/Material/transparent.png} }
				\end{minipage}
				\begin{minipage}[b]{0.8in}
					\centerline{ \includegraphics[width=0.8in, height=.65in]{./Picture/Material/transparent.png} }
				\end{minipage}
				\begin{minipage}[b]{0.8in}
					\centerline{ \includegraphics[width=0.8in, height=.65in]{./Picture/Material/transparent.png} }
				\end{minipage}
				\begin{minipage}[b]{0.8in}
					\centerline{ \includegraphics[width=0.8in, height=.65in]{./Picture/Material/transparent.png} }
				\end{minipage}
				\begin{minipage}[b]{0.6in}
					\centerline{ \includegraphics[width=.45in, height=.6in]{./Picture/Material/down.jpg} }
				\end{minipage}
			\end{figure}
			\vspace{-50pt} 
			\hspace{-65pt}
			\rightline{\tiny{RANSAC\quad}}
			\vspace{15pt} 
			\begin{figure}[htpb]
				\centering
				\begin{minipage}[b]{0.8in}
%					\textbf{\huge{
%							\begin{equation*}
%								\begin{bmatrix}R & t\\ 0 & 1 \end{bmatrix}
%							\end{equation*}
%						}
%					}
					\label{PROBLEM-------cann't show matirx img}
					\centerline{ \includegraphics[width=0.65in, height=.65in]{./Picture/Material/matrix.eps} }
				\end{minipage}
				\begin{minipage}[b]{0.8in}
					\centerline{ \quad}
					\centerline{ \includegraphics[width=0.6in, height=.35in]{./Picture/Material/left.jpg} }
					\centerline{\tiny{变换矩阵}}
				\end{minipage}
				\begin{minipage}[b]{0.8in}
					\centerline{ \includegraphics[width=0.8in, height=.65in]{./Picture/Example/match.png} }
				\end{minipage}
				\begin{minipage}[b]{0.8in}
					\centerline{ \includegraphics[width=0.6in, height=.35in]{./Picture/Material/left.jpg} }
					\centerline{\tiny{ICP}}
				\end{minipage}
				\begin{minipage}[b]{0.8in}
					\centerline{ \includegraphics[width=0.8in, height=.65in]{./Picture/Example/mid.png} }
				\end{minipage}
			\end{figure}
		\end{exampleblock}
	\end{frame}
	\begin{frame}{点云配准}
		\begin{exampleblock}{输入}
			\begin{itemize}		
				\item  区域点云数据: $Points Cloud$
				\item  3D物体点云数据: $Model_{label}$
				\item  最小相似度: $sim$
				\item  最大可接受距离: $mad$
				\item  最大迭代次数: $mi$ 
				\item  接受比例: $an$
			\end{itemize}
		\end{exampleblock}
		\begin{exampleblock}{输出}
			\begin{itemize}
				\item 旋转矩阵: $R$
				\item 平移向量: $\vec t$ 
			\end{itemize}
		\end{exampleblock}
	\end{frame}
	
	\begin{frame}{点云配准}
		\vspace{-9pt}
		\begin{exampleblock}{过程}
		\end{exampleblock}
		\vspace{-20pt}
		\begin{exampleblock}{\small{PART 1: RANSAC}}
			$R\leftarrow dig(1,1,1)$, $\vec t\leftarrow (0,0,0)^T$ 
			\\ \textbf{repeat} 
			\begin{itemize}
				\item $mrs \leftarrow $ Random-Select$(Model_{label})$
				\item $pcrs \leftarrow$ Random-Select$(Points Cloud)$
				\item $similar \leftarrow \{<p1,p2>|Similarity(FPFH_{p1},FPFH_{p2})>sim  \wedge p1 \in mrs \wedge p2 \in pcrs\}$
				\item $R$, $\vec t$ $\leftarrow$ Estimate-TransformMatrix$(similar)$
				\item $consensus \leftarrow \{p| min (distance(R*p+\vec t,q), q\in Point Cloud) < mad \wedge p \in Model_{label} \}$
			\end{itemize}
			\textbf{until} \quad $\frac{|consensus|}{|Point Cloud|} > ar$ or Reach $mi$
		\end{exampleblock}
	\end{frame}
	
	\begin{frame}{点云配准}
		\begin{exampleblock}{过程 (\emph{续} )}
		\end{exampleblock}
		\vspace{-20pt}
		\begin{exampleblock}{\small{PART 2: ICP}}
			${Model'}_{label} = R * Model_{label} + \vec t$
			\\ $R_{icp}$, $\vec t_{icp} \leftarrow Classic$-$ICP({Model'}_{label}, Points Cloud)$
			\\ $R \leftarrow R_{icp} * R$
			\\ $\vec t \leftarrow R_{icp} * \vec t + \vec t_{icp}$
			\\ \textbf{return} $R$, $\vec t$
		\end{exampleblock}
	\end{frame}	
	\subsection{实验结果}
	\begin{frame}{配准效果}
		\begin{exampleblock}{地面实况(Groundtruth)与配准效果}
			\begin{figure}[htpb]
				\centering
				\subfigure[地面实况]{
					\begin{minipage}[b]{1.1in}
						\includegraphics[width=1.1in]{./Picture/Example/groundtruth.eps}
					\end{minipage}
				}
				\subfigure[瓶子(bottle)]{
					\begin{minipage}[b]{1.1in}
						\includegraphics[width=1.1in]{./Picture/Example/bottle-res.eps}
					\end{minipage}
				}
				\subfigure[杯子(cup)]{
					\begin{minipage}[b]{1.1in}
						\includegraphics[width=1.1in]{./Picture/Example/cup-res.eps}
					\end{minipage}
				}
				\subfigure[罐子(can)]{
					\begin{minipage}[b]{1.1in}
						\includegraphics[width=1.1in]{./Picture/Example/can-res.eps}
					\end{minipage}
				}
				\subfigure[茶壶(teapot)]{
					\begin{minipage}[b]{1.1in}
						\includegraphics[width=1.1in]{./Picture/Example/teapot-res.eps}
					\end{minipage}
				}
				\subfigure[盒子(box)]{
					\begin{minipage}[b]{1.1in}
						\includegraphics[width=1.1in]{./Picture/Example/box-res.eps}
					\end{minipage}
				}
				\label{FIG ResultShow} 
			\end{figure}
		\end{exampleblock}
	\end{frame}
	
	\begin{frame}{实验结果}
		\vspace{-8pt}
		\begin{exampleblock}{评价标准}
			\centering{\large{$J = \frac{|P \bigcap G|}{|P \bigcup G|}$}}
			\begin{itemize}
				\item \footnotesize{$P$是预测矩形区域,$G$是地面实况;$|X|$表示矩形$X$的面积}
			\end{itemize}
		\end{exampleblock}
		\vspace{-8pt}
		\begin{exampleblock}{与基于ICP方法的比较结果}
			% 配准效果表
			\begin{table}[htbp]
				\centering
				\begin{tabular}{lcccc}
					\toprule
					\multicolumn{1}{c}{类别}    & \multicolumn{1}{l}{提出的方法} & \multicolumn{1}{l}{经典ICP} & \multicolumn{1}{l}{带法线ICP} & 非线性ICP			 \\
					\midrule
					瓶子(bottle)   & 0.657 & 0.309 & 0.307 & 0.316			 \\
					%\midrule
					盒子(box)    & 0.619 & 0.225 & 0.241 & 0.289			 \\
					%\midrule
					罐子(can)    & 0.597 & 0.445 & 0.372 & 0.329			 \\
					%\midrule
					茶杯(cup)     & 0.695 & 0.042 & 0.132 & 0.177			 \\
					%\midrule
					茶壶(teapot)    & 0.658 & 0.165 & 0.237 & 0.215			 \\
					\bottomrule
				\end{tabular}%
				\label{TAB RegistrationCompare}%
			\end{table}%
		\end{exampleblock}
	\end{frame}	


\section{物体抓取系统的实现}	
	\subsection{坐标转换}
	\begin{frame}{坐标转换}
		\vspace{-24pt}
		\begin{columns}[t]
			\column{0.5\textwidth}
			\begin{exampleblock}{相机标定}
				\vspace{-10pt}
				\begin{equation*}
					\begin{bmatrix} X_P\\  Y_P\\  Z_P \end{bmatrix} = 
					z_c	{\begin{bmatrix} f/dx & 0 & x_{0} \\  0 & f/dy & y_{0} \\  0 & 0 & 1 \end{bmatrix}}^{-1} \begin{bmatrix} u\\  v\\  1 \end{bmatrix}
				\end{equation*}	
				\vspace{-10pt}
				\begin{equation*}
					\begin{bmatrix} X_{P}\\  Y_{P}\\  Z_{P} \end{bmatrix} =
					R \times \begin{bmatrix} X_{C}\\  Y_{C}\\  Z_{C} \end{bmatrix} +  \vec{t}
				\end{equation*}
			\end{exampleblock}
			\vspace{-10pt}
			\begin{exampleblock}{机械臂标定}
				\begin{equation*}
				\begin{bmatrix} X_{A}\\  Y_{A}\\  Z_{A}\end{bmatrix} =
				\begin{bmatrix} X_C\\  Y_C\\  Z_C \end{bmatrix} +
				\begin{bmatrix} X_{0}\\  Y_{0}\\  Z_{0}\end{bmatrix} 
				\end{equation*}
			\end{exampleblock}
		
			\column{0.36\textwidth}
			\begin{exampleblock}{利用棋盘进行标定}
				\includegraphics[width=\textwidth]{./Picture/Calibration/corner.png}
				\\
				
				\includegraphics[width=\textwidth]{./Picture/Calibration/InitPoint.png}
			\end{exampleblock}
		\end{columns}
	\end{frame}	

	\subsection{抓取演示}
	
	\begin{frame}{\href{./Video/demo-grasping.mp4}{抓取演示}}
		\includegraphics[width=\textwidth,height=0.85\textheight]{./Picture/Material/cover.png}
	\end{frame}	


\section{总结与展望}	
\subsection{总结}
\begin{frame}{本文贡献}
	\begin{exampleblock}{RGB-D快速候选区域筛选}
		\begin{itemize}
			\item 基于深度信息的区域增长法对场景区域进行分割
			\item 基于颜色信息的HOG特征加SVM分类器对区域进行判别
		\end{itemize}
	\end{exampleblock}	
	\begin{exampleblock}{点云配准抓取定位}
		\begin{itemize}
			\item 结合RANSAC与ICP将部分点云与整体点云配准以实现抓取定位
		\end{itemize}
	\end{exampleblock}
	\begin{exampleblock}{系统实现}
		\begin{itemize}
			\item 利用黑白棋盘进行深度相机与机械臂的标定
		\end{itemize}
	\end{exampleblock}
\end{frame}	

\subsection{展望}
\begin{frame}{本文不足}
	\begin{exampleblock}{预处理过程繁杂}
		\begin{itemize}
			\item 三维物体建模
			\item 二维颜色数据采集
		\end{itemize}
	\end{exampleblock}
	\begin{exampleblock}{可扩展性稍弱}
		\begin{itemize}
			\item 添加新物体时,要做两步预处理
		\end{itemize}
	\end{exampleblock} 
	 \begin{exampleblock}{抓取精度}
	 	\begin{itemize}
	 		\item 较为细小的物体
	 		\item 无点云物体
	 	\end{itemize}
	 \end{exampleblock}
 
\end{frame}	

\begin{frame}{展望}
	\begin{exampleblock}{进一步研究思路}
		\begin{itemize}
			\item 相似三维结构聚类
			\item 几何学上的信息推断
			\item 利用学习模型进行预测
		\end{itemize}
	\end{exampleblock}
\end{frame}


%%%%%%%%%%%%%%%%%%%%%%%%%%%%%%%%%%%%%%%%%%%%%%%%%%%%%%%%%%%%%%%%%%%%%%%%%
\section{}
%%%%%%%%%%%%%%%%%%%%%%%%%%%%%%%%%%%%%%%%%%%%%%%%%%%%%%%%%%%%%%%%%%%%%%%%%
\begin{frame}{科研成果}
	\begin{exampleblock}{发表论文}
		\textbf{\underline{Yiyang Zhou}}, Wenhai Wang, Wenjie Guan, Yirui Wu, Heng Lai, Tong Lu, Min Cai. Visual Robotic Object Grasping Through Combining RGB-D Data and 3D Meshes. The 23rd International Conference on MultiMedia Modeling (MMM’17), Reykjavik, Iceland, accepted for Oral Presentation, pp. 404-415, 2017 (CCF-C类会议)
	\end{exampleblock}
	\begin{exampleblock}{申请专利}
		路通,\textbf{\underline{周逸徉}},蔡敏,“基于点云配准的物体抓取区域定位方法”,国家专利,申请号: 201610998429.3
	\end{exampleblock}
\end{frame}

\begin{frame}[plain]
	\thispagestyle{empty}
	\begin{columns}
		\begin{column}{\paperwidth}
			\includegraphics[width=\paperwidth,height=\paperheight]{./Picture/Material/3Q.png}
		\end{column}
	\end{columns}
\end{frame}
\begin{frame}[plain]
	\thispagestyle{empty}
	\begin{columns}
		\begin{column}{\paperwidth}
			\includegraphics[width=\paperwidth,height=\paperheight]{./Picture/Material/QA.jpg}
		\end{column}
	\end{columns}
\end{frame}
%%%%%%%%%%%%%%%%%%%%%%%%%%%%%%%%%%%%%%%%%%%%%%%%%%%%%%%%%%%%%%%%%%%%%%%%%
\end{document}