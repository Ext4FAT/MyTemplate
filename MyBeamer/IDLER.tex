%%%%%%%%%%%%%%%%%%%%%%%%%%%%%%%%%%%%%%%%%%%%%%%%%%%%%%%%%%%%%%%%%%%%%%%%%%%%%
% 11/18/2016
% re-edited by Yiyang ZHOU
%
%
%%%%%%%%%%%%%%%%%%%%%%%%%%%%%%%%%%%%%%%%%%%%%%%%%%%%%%%%%%%%%%%%%%%%%%%%%%%%%
%%%%%%%%%%%%%%%%%%%%%%%%%%%%%%%%%%%%%%%%%%%%%%%%%%%%%%%%%%%%%%%%%%%%%%%%%%%%%
% 01/08/2011
% re-edited by Zhiqiang Lin
%
%
%%%%%%%%%%%%%%%%%%%%%%%%%%%%%%%%%%%%%%%%%%%%%%%%%%%%%%%%%%%%%%%%%%%%%%%%%%%%%
%%%%%%%%%%%%%%%%%%%%%%%%%%%%%%%%%%%%%%%%%%%%%%%%%%%%%%%%%%%%%%%%%%%%%%%%%%%%%
% 26/05/2010
% edited by Bill Lampos
%
% Feel free to use (copy) the structure (latex formatting source code)
% but not the content of this document.
%
%%%%%%%%%%%%%%%%%%%%%%%%%%%%%%%%%%%%%%%%%%%%%%%%%%%%%%%%%%%%%%%%%%%%%%%%%%%%%
\documentclass[xcolor=table,compress,blue]{beamer}
\mode<presentation>

\usetheme{Boadilla}
%\usetheme{Warsaw}
%\usetheme{CambridgeUS}
%\usetheme{JuanLesPins} % A good theme which eliminates the bottom lines
%\usetheme{Singapore}%Good for release to print

% other themes: AnnArbor, Antibes, Bergen, Berkeley, Berlin, Boadilla, boxes, CambridgeUS, Copenhagen, Darmstadt, default, Dresden, Frankfurt, Goettingen,
% Hannover, Ilmenau, JuanLesPins, Luebeck, Madrid, Maloe, Marburg, Montpellier, PaloAlto, Pittsburg, Rochester, Singapore, Szeged, classic

%\usecolortheme{whale}
% color themes: albatross, beaver, beetle, crane, default, dolphin, dov, fly, lily, orchid, rose, seagull, seahorse, sidebartab, structure, whale, wolverine

%\usefonttheme{serif}
%\usefonttheme{professionalfonts}
% font themes: default, professionalfonts, serif, structurebold, structureitalicserif, structuresmallcapsserif


% pdf is displayed in full screen mode automatically
%\hypersetup{pdfpagemode=FullScreen}


% define your own colours:
\definecolor{Red}{rgb}{1,0,0}
\definecolor{Blue}{rgb}{0,0,1}
\definecolor{Green}{rgb}{0,1,0}
\definecolor{magenta}{rgb}{1,0,.6}
\definecolor{lightblue}{rgb}{0,.5,1}
\definecolor{lightpurple}{rgb}{.6,.4,1}
\definecolor{gold}{rgb}{.6,.5,0}
\definecolor{orange}{rgb}{1,0.4,0}
\definecolor{hotpink}{rgb}{1,0,0.5}
\definecolor{newcolor2}{rgb}{.5,.3,.5}
\definecolor{newcolor}{rgb}{0,.3,1}
\definecolor{newcolor3}{rgb}{1,0,.35}
\definecolor{darkgreen1}{rgb}{0, .35, 0}
\definecolor{darkgreen}{rgb}{0, .6, 0}
\definecolor{darkred}{rgb}{.75,0,0}

\xdefinecolor{olive}{cmyk}{0.64,0,0.95,0.4}
\xdefinecolor{purpleish}{cmyk}{0.75,0.75,0,0}

% \usepackage{beamerinnertheme_______}
% inner themes include circles, default, inmargin, rectangles, rounded

%\usepackage{beamerouterthemesmoothbars}
% outer themes include default, infolines, miniframes, shadow, sidebar, smoothbars, smoothtree, split, tree

\useoutertheme[subsection=false]{smoothbars}

% to have the same footer on all slides

\setbeamertemplate{footline}{}
\setbeamercolor{background canvas}{bg=}
%\setbeamertemplate{background}{\includegraphics{Imagine.jpg}}
%\setbeamertemplate{footline}[text line]{xxx xxx xxx}
%\setbeamertemplate{footline}[text line]{} % or empty footer


% include packages
\usepackage{float} 
\usepackage{algorithmic}
\usepackage{algorithm}
\usepackage{subfigure}
\usepackage{multicol}
\usepackage{amsmath}
\usepackage{graphicx}
\usepackage{epstopdf}
\usepackage{wallpaper}

%\usepackage{epsfig}
%\usepackage{movie15}
%\usepackage[all,knot]{xy}

%\xyoption{arc}
\usepackage{url}
%\usepackage{multimedia}
\usepackage{hyperref}
\usepackage{colortbl}
\usepackage{fancyhdr}
\usepackage{eso-pic}
\usepackage{shortcuts}


%\usepackage{xcolor=pdftex}


% Standard packages

\usepackage[english]{babel}
\usepackage[latin1]{inputenc}
\usepackage{times}
\usepackage[T1]{fontenc}
\usepackage{animate}

%%%
\usepackage{tikz}
\usetikzlibrary{arrows, shapes}
\tikzstyle{block}=[draw opacity=0.7,line width=1.4cm]
%%%

\usepackage{booktabs}
%\usepackage{ctable}
%\usepackage{multirow}

% my custom config
\usepackage{amsmath}
\DeclareMathOperator*{\argmax}{\argmax}
\setcounter{tocdepth}{3}
\renewcommand{\algorithmicrequire}{\textbf{Input:}}
\renewcommand{\algorithmicensure}{\textbf{Output:}}


%%%%%%%%%%%%%%%%%%%%%%%%%%%%%%%%%%%%%%%%%%%%%%%%%%%%%%%%%%%%%%%%%%%%%%%%%
%%% store graphics in a box, background
\newsavebox{\mygraphic}
\sbox{\mygraphic}{%
	\includegraphics[keepaspectratio,height=\textheight,width=\textwidth]{./Picture/Material/MMM2017.png}}
\AddToShipoutPicture{
	\put(20,40){\usebox{\mygraphic}}
	%put(-2.2,6){\makebox(-2.2,-6)[bl]{\usebox{\mygraphic}}}
}
%%%%%%%%%%%%%%%%%%%%%%%%%%%%%%%%%%%%%%%%%%%%%%%%%%%%%%%%%%%%%%%%%%%%%%%%%


%%%%%%%%%%%%%%%%%%%%%%%%%%%%%%%%%%%%%%%%%%%%%%%%%%%%%%%%%%%%%%%%%%%%%%%%%
\AtBeginSection[] {
  \begin{frame}[plain]
    \frametitle{Outline}
    \tableofcontents[currentsection]
  \end{frame}
  \addtocounter{framenumber}{-1}
}

\iffalse
\AtBeginSubsection[] {
  \begin{frame}[plain]
   \frametitle{Outline}
    \tableofcontents[currentsubsection]
    \addtocounter{framenumber}{-1}
  \end{frame}
}
\fi
%%%%%%%%%%%%%%%%%%%%%%%%%%%%%%%%%%%%%%%%%%%%%%%%%%%%%%%%%%%%%%%%%%%%%%%%%

%\subtitle{\Large{\textbf{Lecture 1: Course Overview}}}

\newcommand{\spacetwo}{\textcolor{white}{\\sp\\ace}}
\newcommand{\spaceone}{\textcolor{white}{sp\\ace}}

\title{\LARGE{\textbf{Visual robotic object grasping through combining RGB-D data and 3D mesh}}}
%\titlegraphic{\includegraphics[height=\paperheight]{QA.jpg}}
\author{\textbf{Yiyang Zhou$^{1}$} \and Wenhai Wang$^{1}$ \and Wenjie Guan$^{1}$ \and Yirui Wu$^{2}$ \and Heng Lai$^{1}$ \and Tong Lu$^{1}$%
	\and Min Cai$^{3}$}
\institute{ 
	$^{1}$ State Key Laboratory for Novel Software Technology \\ 
	Nanjing University, Nanjing, China \\
	$^{2}$ College of Computer and Information, Hohai University, Nanjing, China \\
	$^{3}$ Riseauto Intelligent Tech, Beijing, China
}
\date{\tiny{}}

%\logo{\includegraphics[width=0.55cm,height=0.35cm]{nju.jpg}}



%\author{\large Mao Bing\spaceone}
%\institute{Department of Computer Science \par NanJing University }
\titlegraphic{\quad \quad \includegraphics[height=1.5cm]{./Picture/Logo/nju.jpg}\quad \quad \includegraphics[height=1.6cm]{./Picture/Logo/hohai.png}\includegraphics[height=1.6cm]{./Picture/Logo/riseauto.png}}


\begin{document}

\frame{
    \titlepage
    \thispagestyle{empty}
}
\begin{frame}{Outline}
  \tableofcontents
\end{frame}

%%%%%%%%%%%%%%%%%%%%%%%%%%%%%%%%%%%%%%%%%%%%%%%%%%%%%%%%%%%%%%%%%%%%%%%%%
\section{INTRODUCTION}
%%%%%%%%%%%%%%%%%%%%%%%%%%%%%%%%%%%%%%%%%%%%%%%%%%%%%%%%%%%%%%%%%%%%%%%%%
\subsection{}
\begin{frame}{INTRODUCTION}%{BACKGROUND}
	\begin{exampleblock}{Challenges in Grasping region localization}
		Robotic grasping driven by cameras is challenging due to
		a lot inherent difficulties brought by 2D vision based systems, especially inaccuracy in determining sizes, distance and orientations for visual objects. 
	\end{exampleblock}
	\pause
	\begin{exampleblock}{RGB-D camera widely application}
		 Much more attentions have been caught in RGB-D camera
		 based robotics explorations in the multimedia community today, such as indoor environment 3D modeling, self-localization and navigation. 
	\end{exampleblock}
	\pause
	\begin{exampleblock}{Why not have a try with RGB-D camera}
		Due to the inherent limitations and difficulties, meanwhile with widely application of RGB-D, shall we propose a low-price and easy-to-use framework to solve this task? 
	\end{exampleblock}
\end{frame}

\iffalse
\begin{frame}{RELATED WORK}
	\begin{exampleblock}{BACKGROUND}
		
	\end{exampleblock}
\end{frame}
\fi
%%%%%%%%%%%%%%%%%%%%%%%%%%%%%%%%%%%%%%%%%%%%%%%%%%%%%%%%%%%%%%%%%%%%%%%%%



 
%%%%%%%%%%%%%%%%%%%%%%%%%%%%%%%%%%%%%%%%%%%%%%%%%%%%%%%%%%%%%%%%%%%%%%%%%
\section{FRAMEWORK}
%%%%%%%%%%%%%%%%%%%%%%%%%%%%%%%%%%%%%%%%%%%%%%%%%%%%%%%%%%%%%%%%%%%%%%%%%
\subsection{}
\begin{frame}{Framework}
	\begin{exampleblock}{Proposed framework for robotic grasping}
		
	\centering{
	\includegraphics[height=2.16in,width=4.68in]{./Picture/Example/flow.eps}
	}
	
	\end{exampleblock}
\end{frame}
%%%%%%%%%%%%%%%%%%%%%%%%%%%%%%%%%%%%%%%%%%%%%%%%%%%%%%%%%%%%%%%%%%%%%%%%%



%%%%%%%%%%%%%%%%%%%%%%%%%%%%%%%%%%%%%%%%%%%%%%%%%%%%%%%%%%%%%%%%%%%%%%%%%
\section{PRE-PROCESSING}
%%%%%%%%%%%%%%%%%%%%%%%%%%%%%%%%%%%%%%%%%%%%%%%%%%%%%%%%%%%%%%%%%%%%%%%%%
\subsection{Object detector training}
\begin{frame}{Object detector training}
	\vspace{-28pt}
	\begin{exampleblock}{Collect 2D object image as training examples}
		\begin{figure}[htpb]
			\centering
			\begin{minipage}[b]{0.8in}
				\centerline{\includegraphics[width=0.8in, height=.8in]{./Picture/HOG/11.png}}
			\end{minipage}
			\begin{minipage}[b]{0.8in}
				\centerline{\includegraphics[width=0.8in, height=.8in]{./Picture/HOG/22.png}}
			\end{minipage}
			\begin{minipage}[b]{0.8in}
				\centerline{\includegraphics[width=0.8in, height=.8in]{./Picture/HOG/33.png}}
			\end{minipage}
			\begin{minipage}[b]{0.8in}
				\centerline{\includegraphics[width=0.8in, height=.8in]{./Picture/HOG/44.png}}
			\end{minipage}
			\begin{minipage}[b]{0.8in}
				\centerline{\includegraphics[width=0.8in, height=.8in]{./Picture/HOG/55.png}}
			\end{minipage}
			\label{fig:Models} %% label for entire figure
		\end{figure}
	\end{exampleblock}
	\vspace{-16pt}
	\begin{exampleblock}{Scale to the same size and extract HOG feature}
		\begin{figure}[htpb]
			\centering
			\begin{minipage}[b]{0.8in}
				\centerline{\includegraphics[width=0.8in, height=.8in]{./Picture/HOG/1.png}}
			\end{minipage}
			\begin{minipage}[b]{0.8in}
				\centerline{\includegraphics[width=0.8in, height=.8in]{./Picture/HOG/2.png}}
			\end{minipage}
			\begin{minipage}[b]{0.8in}
				\centerline{\includegraphics[width=0.8in, height=.8in]{./Picture/HOG/3.png}}
			\end{minipage}
			\begin{minipage}[b]{0.8in}
				\centerline{\includegraphics[width=0.8in, height=.8in]{./Picture/HOG/4.png}}
			\end{minipage}
			\begin{minipage}[b]{0.8in}
				\centerline{\includegraphics[width=0.8in, height=.8in]{./Picture/HOG/5.png}}
			\end{minipage}
			\label{fig:Models} %% label for entire figure
		\end{figure}
	\end{exampleblock}
	\vspace{-16pt}
	\begin{exampleblock}{Train SVM classifier with One Vs Rest strategy} 

	\end{exampleblock}
\end{frame}
 
\subsection{Grasping regions definition}
\begin{frame}{Grasping regions definition}
	\begin{exampleblock}{Build models with 3D scanner}
		\centering{
			\includegraphics[width=4.2in, height=1.05in]{./Picture/Example/models.png}	
		}
	\end{exampleblock}
	\begin{exampleblock}{Mark grasping regions on 3D meshes}
		\begin{figure}[htpb]
			\centering
			\begin{minipage}[b]{0.8in}
				\centerline{\includegraphics[width=0.8in, height=.85in]{./Picture/Example/cup.eps}}
				\centerline{\small{1: Cup}}
			\end{minipage}
			\begin{minipage}[b]{0.8in}
				\centerline{\includegraphics[width=0.8in, height=.85in]{./Picture/Example/can.eps}}
				\centerline{\small{2: Can}}
			\end{minipage}
			\begin{minipage}[b]{0.8in}
				\centerline{\includegraphics[width=0.8in, height=.85in]{./Picture/Example/pot.eps}}
				\centerline{\small{3: Teapot}}
			\end{minipage}
			\begin{minipage}[b]{0.8in}
				\centerline{\includegraphics[width=0.8in, height=.85in]{./Picture/Example/box.eps}}
				\centerline{\small{4: Box}}
			\end{minipage}
			\begin{minipage}[b]{0.8in}
				\centerline{\includegraphics[width=0.8in, height=.85in]{./Picture/Example/bottle.eps}}
				\centerline{\small{5: Bottle}}
			\end{minipage}
			\label{fig:Models} %% label for entire figure
		\end{figure}
	\end{exampleblock}
\end{frame}
%%%%%%%%%%%%%%%%%%%%%%%%%%%%%%%%%%%%%%%%%%%%%%%%%%%%%%%%%%%%%%%%%%%%%%%%%



%%%%%%%%%%%%%%%%%%%%%%%%%%%%%%%%%%%%%%%%%%%%%%%%%%%%%%%%%%%%%%%%%%%%%%%%%
\section{GRASPING REGION LOCALIZATION}
%%%%%%%%%%%%%%%%%%%%%%%%%%%%%%%%%%%%%%%%%%%%%%%%%%%%%%%%%%%%%%%%%%%%%%%%%
\subsection{Segmentation}
\label{ALG-SEG}
\tikzstyle{every picture}+=[remember picture] %a global declaration
\begin{frame}{Segmentation}
	\begin{exampleblock}{Flow}
		\begin{figure}[htpb]
			\centering
			\begin{minipage}[b]{0.8in}
				\centerline{ \includegraphics[width=0.8in, height=.65in]{./Picture/Example/depth.eps} }
			\end{minipage}
			\begin{minipage}[b]{0.8in}
				\centerline{ \includegraphics[width=0.6in, height=.35in]{./Picture/Material/arrow.jpg} }
				\centerline{\tiny{Region growing}}
			\end{minipage}
			\begin{minipage}[b]{0.8in}
				\centerline{ \includegraphics[width=0.8in, height=.65in]{./Picture/Example/merge.eps} }
			\end{minipage}
			\begin{minipage}[b]{0.8in}
				\centerline{ \includegraphics[width=0.6in, height=.35in]{./Picture/Material/arrow.jpg} }
				\centerline{\tiny{Merging}}
			\end{minipage}
			\begin{minipage}[b]{0.8in}
				\centerline{ \includegraphics[width=0.8in, height=.65in]{./Picture/Example/seg.eps} }
			\end{minipage}
		\end{figure}
	\end{exampleblock}
	\begin{exampleblock}{Input}
		\begin{itemize}		
			\item  Depth data: $Depth$
			\item  Number of Main Segmentations: ${topk}$
		\end{itemize}
	\end{exampleblock}
	\begin{exampleblock}{Output}
		\begin{itemize}		
			\item  Main Segmentations: ${MS}$
		\end{itemize}
	\end{exampleblock}
\end{frame}
\begin{frame}{Segmentation}
	\begin{exampleblock}{Process}
	\end{exampleblock}
	\vspace{-15pt}
	\begin{exampleblock}{\small{Rough Segment}}
		\begin{itemize}
			\item $<BCR, DCR>$ $\leftarrow$ Region-Growing($Depth$)
			\item Sort $DCR$ according to the number of points
			\item Select $topk$ segmentations as $MS$ from $DCR$
		\end{itemize}
	\end{exampleblock}
	\vspace{-15pt}
	\begin{exampleblock}{\small{Merge}}
		\textbf{for} {$seg \in  (DCR - MS) \cup BCR$}
		\\ \quad \textbf{for} {$p \in seg$}
%		\begin{itemize}
		\\ \quad \quad $CS \leftarrow \{ms|ms \in MS \wedge p\ in\ ConvexHull(ms)\}$
		\\ \quad \quad $SEG \leftarrow \mathop{\arg\min}_{ms}(distance(ms), ms\in CS)$	
		\\ \quad \quad $SEG \leftarrow SEG \cup \{p\}$
%		\end{itemize}
  \\ \quad \textbf{end for}
		\\ \textbf{end for} 
		\\ \textbf{return} $MS$
	\end{exampleblock}

\end{frame}


\subsection{Classification}
\begin{frame}{Classification}
	\begin{exampleblock}{Flow}
		\begin{figure}[htpb]
			\centering
			\begin{minipage}[b]{0.8in}
				\centerline{ \includegraphics[width=0.8in, height=.65in]{./Picture/Example/color.eps} }
			\end{minipage}
			\begin{minipage}[b]{0.8in}
				\centerline{ \includegraphics[width=0.6in, height=.35in]{./Picture/Material/arrow.jpg} }
				\centerline{\tiny{Bounding box}}
			\end{minipage}
			\begin{minipage}[b]{0.8in}
				\centerline{ \includegraphics[width=0.8in, height=.65in]{./Picture/Example/regions.eps} }
			\end{minipage}
			\begin{minipage}[b]{0.8in}
				\centerline{ \includegraphics[width=0.6in, height=.35in]{./Picture/Material/arrow.jpg} }
				\centerline{\tiny{Filtering}}
			\end{minipage}
			\begin{minipage}[b]{0.8in}
				\centerline{ \includegraphics[width=0.8in, height=.65in]{./Picture/Example/classification.eps} }
			\end{minipage}
		\end{figure}
	\end{exampleblock}
	\begin{exampleblock}{Input}
		\begin{itemize}		
			\item Color data: $Color$	
			\item Main Segmentations: $MS$ 
			\item Object detector: ${Classifier}_{obj}$
		\end{itemize}
	\end{exampleblock}
	\begin{exampleblock}{Output}
		\begin{itemize}
			\item Object Regions with label: $Regions$
		\end{itemize}
	\end{exampleblock}
\end{frame}
\begin{frame}{Classification}
	\begin{exampleblock}{Process}
		{$Rect \leftarrow \{rect|rect \leftarrow BoundBox(ConvexHull(ms))\}$}
		\\ {$Candidate \leftarrow \emptyset$}
		\\ \textbf{for} {$rect \in Rect$}
		\\ \quad {${Region}_{color} \leftarrow$ Extract-Region$(Color, rect)$}
		%\\ \quad {${Region}_{64\times64} \leftarrow$ Resize$({Region}_{color})$}
		\\ \quad {${feature}_{HOG} \leftarrow$ Calculate-HOG$({Region}_{color})$}
		\\ \quad {$label \leftarrow$ Classify$({Classifier}_{obj}, {feature}_{HOG})$}
		\\ \quad {$Candidate \leftarrow \{rect,label\} \cup Candidate$}
		\\ \textbf{end for}
		\\ {$Regions \leftarrow \{<r,l>|<r,l> \in Candidate \wedge l \neq 0\}$}
		\\ \textbf{return} $Regions$
	\end{exampleblock}
\end{frame}


\subsection{Registration}
\begin{frame}{Registration}
	\begin{exampleblock}{Flow}
		\begin{figure}[htpb]
			\centering
			\begin{minipage}[b]{0.8in}
				\centerline{ \includegraphics[width=0.8in, height=.65in]{./Picture/Material/realsense.jpg} }
			\end{minipage}
			\begin{minipage}[b]{0.8in}
				\centerline{ \quad}
				\centerline{ \includegraphics[width=0.6in, height=.35in]{./Picture/Material/arrow.jpg} }
				\centerline{\tiny{Generate Points Cloud}}
			\end{minipage}
			\begin{minipage}[b]{0.8in}
				\centerline{ \includegraphics[width=0.8in, height=.65in]{./Picture/Example/pointcloud.eps} }
			\end{minipage}
			\begin{minipage}[b]{0.8in}
				\centerline{ \includegraphics[width=0.6in, height=.35in]{./Picture/Material/arrow.jpg} }
				\centerline{\tiny{Loading model}}
			\end{minipage}
			\begin{minipage}[b]{0.8in}
				\centerline{ \includegraphics[width=0.8in, height=.65in]{./Picture/Example/before-match.png} }
			\end{minipage}
		\end{figure}
		\vspace{-25pt}
		\begin{figure}[htpb]
			\centering
			\begin{minipage}[b]{0.8in}
				\centerline{ \includegraphics[width=0.8in, height=.65in]{./Picture/Material/transparent.png} }
			\end{minipage}
			\begin{minipage}[b]{0.8in}
				\centerline{ \includegraphics[width=0.8in, height=.65in]{./Picture/Material/transparent.png} }
			\end{minipage}
			\begin{minipage}[b]{0.8in}
				\centerline{ \includegraphics[width=0.8in, height=.65in]{./Picture/Material/transparent.png} }
			\end{minipage}
			\begin{minipage}[b]{0.8in}
				\centerline{ \includegraphics[width=0.8in, height=.65in]{./Picture/Material/transparent.png} }
			\end{minipage}
			\begin{minipage}[b]{0.6in}
				\centerline{ \includegraphics[width=.45in, height=.6in]{./Picture/Material/down.jpg} }
			\end{minipage}
		\end{figure}
		\vspace{-50pt} 
		\hspace{-65pt}
		\rightline{\tiny{RANSAC}}
		\vspace{15pt} 
		\begin{figure}[htpb]
			\centering
			\begin{minipage}[b]{0.8in}
				\centerline{ \includegraphics[width=0.65in, height=.65in]{./Picture/Material/matrix.png} }
			\end{minipage}
			\begin{minipage}[b]{0.8in}
				\centerline{ \quad}
				\centerline{ \includegraphics[width=0.6in, height=.35in]{./Picture/Material/left.jpg} }
				\centerline{\tiny{Transformation matrix}}
			\end{minipage}
			\begin{minipage}[b]{0.8in}
				\centerline{ \includegraphics[width=0.8in, height=.65in]{./Picture/Example/match.png} }
			\end{minipage}
			\begin{minipage}[b]{0.8in}
				\centerline{ \includegraphics[width=0.6in, height=.35in]{./Picture/Material/left.jpg} }
				\centerline{\tiny{ICP}}
			\end{minipage}
			\begin{minipage}[b]{0.8in}
				\centerline{ \includegraphics[width=0.8in, height=.65in]{./Picture/Example/mid.png} }
			\end{minipage}
		\end{figure}
	\end{exampleblock}
\end{frame}
\begin{frame}{Registration}
	\begin{exampleblock}{Input}
		\begin{itemize}		
			\item  Points Cloud Region: $Points Cloud$
			\item  3D Object Models: $Model_{label}$
			\item  Min Similarity: $sim$
			\item  Max Acceptable Distance: $mad$
			\item  Max Iterations: $mi$ 
			\item  Acceptable Number: $an$
		\end{itemize}
	\end{exampleblock}
	\begin{exampleblock}{Output}
		\begin{itemize}
			\item Rotate Matrix: $R$
			\item Translation Vector: $\vec t$ 
		\end{itemize}
	\end{exampleblock}
\end{frame}

\begin{frame}{Registration}
	\begin{exampleblock}{Process}
	\end{exampleblock}
	\vspace{-15pt}
	\begin{exampleblock}{\footnotesize{RANSAC}}
		$R\leftarrow dig(1,1,1)$, $\vec t\leftarrow (0,0,0)^T$ 
		\\ \textbf{repeat} 
		\begin{itemize}
		\item $mrs \leftarrow $ Random-Select$(Model_{label})$
		\item $pcrs \leftarrow$ Random-Select$(Points Cloud)$
		\item $similar$-$pairs \leftarrow \{<p1,p2>|Similarity(FPFH_{p1},FPFH_{p2})>sim  \wedge p1 \in mrs \wedge p2 \in pcrs\}$
		\item $R$, $\vec t$ $\leftarrow$ Estimate-TransformMatrix$(similar$-$pairs)$
		\item $inliers \leftarrow \{p| <p,*> \in $ $similar$-$pairs\}$
		\item $alsoinliers \leftarrow \{p| min (distance(R*p+\vec t,q), q\in pcrs) < mad \wedge p \in mrs-inliers \}$
		\end{itemize}
		\textbf{until} \quad $|alsoinliers| > an$ or Reach $mi$
	\end{exampleblock}
\end{frame}
\begin{frame}{Registration}
	\begin{exampleblock}{Process (\emph{con't} )}
	\end{exampleblock}
	\vspace{-15pt}
	\begin{exampleblock}{\footnotesize{ICP}}
		$R_{icp}$, $\vec t_{icp} \leftarrow Classic$-$ICP(R * Model_{label}, Points Cloud)$
		\\ $R \leftarrow R_{icp} * R$
		\\ $\vec t \leftarrow R_{icp} * \vec t + \vec t_{icp}$
		\\ \textbf{return} $R$, $\vec t$
	\end{exampleblock}
\end{frame}	


\subsection{Localization}
\begin{frame}{Localization}
	\begin{exampleblock}{Grasping region localization}
		\centering{
			\includegraphics[width=2.7in]{./Picture/Example/reflect.eps}
		}
		\begin{itemize}
		\item \footnotesize{Yellow Region: predicated regions for grasping }
		\item \footnotesize{Purple Region: aligned model points region }
		\item \footnotesize{Blue rectangle: bounding box of grasping region }
		\end{itemize}
	\end{exampleblock}
\end{frame}
%%%%%%%%%%%%%%%%%%%%%%%%%%%%%%%%%%%%%%%%%%%%%%%%%%%%%%%%%%%%%%%%%%%%%%%%%



%%%%%%%%%%%%%%%%%%%%%%%%%%%%%%%%%%%%%%%%%%%%%%%%%%%%%%%%%%%%%%%%%%%%%%%%%
\section{EXPERIMENT}
%%%%%%%%%%%%%%%%%%%%%%%%%%%%%%%%%%%%%%%%%%%%%%%%%%%%%%%%%%%%%%%%%%%%%%%%%
\subsection{Dataset}
\begin{frame}{Dataset}
	\begin{exampleblock}{Five typical desktop things from different gestures}
		\begin{figure}[htpb]
			\centering
			\begin{minipage}[b]{0.65in}
				\centerline{ \includegraphics[width=0.6in, height=.45in]{./Picture/Dataset/bottle/bottle(1).png} }
			\end{minipage}
			\begin{minipage}[b]{0.65in}
				\centerline{ \includegraphics[width=0.6in, height=.45in]{./Picture/Dataset/bottle/bottle(2).png} }
			\end{minipage}
			\begin{minipage}[b]{0.65in}
				\centerline{ \includegraphics[width=0.6in, height=.45in]{./Picture/Dataset/bottle/bottle(3).png} }
			\end{minipage}
			\begin{minipage}[b]{0.65in}
				\centerline{ \includegraphics[width=0.6in, height=.45in]{./Picture/Dataset/bottle/bottle(4).png} }
			\end{minipage}
			\begin{minipage}[b]{0.65in}
				\centerline{ \includegraphics[width=0.6in, height=.45in]{./Picture/Dataset/bottle/bottle(5).png} }
			\end{minipage}
			\begin{minipage}[b]{0.65in}
				\centerline{ \includegraphics[width=0.6in, height=.45in]{./Picture/Dataset/bottle/bottle(6).png} }
			\end{minipage}
			\begin{minipage}[b]{0.65in}
				\centerline{ \includegraphics[width=0.6in, height=.45in]{./Picture/Dataset/box/box(1).png} }
			\end{minipage}
			\begin{minipage}[b]{0.65in}
				\centerline{ \includegraphics[width=0.6in, height=.45in]{./Picture/Dataset/box/box(2).png} }
			\end{minipage}
			\begin{minipage}[b]{0.65in}
				\centerline{ \includegraphics[width=0.6in, height=.45in]{./Picture/Dataset/box/box(3).png} }
			\end{minipage}
			\begin{minipage}[b]{0.65in}
				\centerline{ \includegraphics[width=0.6in, height=.45in]{./Picture/Dataset/box/box(4).png} }
			\end{minipage}
			\begin{minipage}[b]{0.65in}
				\centerline{ \includegraphics[width=0.6in, height=.45in]{./Picture/Dataset/box/box(5).png} }
			\end{minipage}
			\begin{minipage}[b]{0.65in}
				\centerline{ \includegraphics[width=0.6in, height=.45in]{./Picture/Dataset/box/box(6).png} }
			\end{minipage}
			\begin{minipage}[b]{0.65in}
				\centerline{ \includegraphics[width=0.6in, height=.45in]{./Picture/Dataset/can/can(1).png} }
			\end{minipage}
			\begin{minipage}[b]{0.65in}
				\centerline{ \includegraphics[width=0.6in, height=.45in]{./Picture/Dataset/can/can(2).png} }
			\end{minipage}
			\begin{minipage}[b]{0.65in}
				\centerline{ \includegraphics[width=0.6in, height=.45in]{./Picture/Dataset/can/can(3).png} }
			\end{minipage}
			\begin{minipage}[b]{0.65in}
				\centerline{ \includegraphics[width=0.6in, height=.45in]{./Picture/Dataset/can/can(4).png} }
			\end{minipage}
			\begin{minipage}[b]{0.65in}
				\centerline{ \includegraphics[width=0.6in, height=.45in]{./Picture/Dataset/can/can(5).png} }
			\end{minipage}
			\begin{minipage}[b]{0.65in}
				\centerline{ \includegraphics[width=0.6in, height=.45in]{./Picture/Dataset/can/can(6).png} }
			\end{minipage}
			\begin{minipage}[b]{0.65in}
				\centerline{ \includegraphics[width=0.6in, height=.45in]{./Picture/Dataset/cup/cup(1).png} }
			\end{minipage}
			\begin{minipage}[b]{0.65in}
				\centerline{ \includegraphics[width=0.6in, height=.45in]{./Picture/Dataset/cup/cup(2).png} }
			\end{minipage}
			\begin{minipage}[b]{0.65in}
				\centerline{ \includegraphics[width=0.6in, height=.45in]{./Picture/Dataset/cup/cup(3).png} }
			\end{minipage}
			\begin{minipage}[b]{0.65in}
				\centerline{ \includegraphics[width=0.6in, height=.45in]{./Picture/Dataset/cup/cup(4).png} }
			\end{minipage}
			\begin{minipage}[b]{0.65in}
				\centerline{ \includegraphics[width=0.6in, height=.45in]{./Picture/Dataset/cup/cup(5).png} }
			\end{minipage}
			\begin{minipage}[b]{0.65in}
				\centerline{ \includegraphics[width=0.6in, height=.45in]{./Picture/Dataset/cup/cup(6).png} }
			\end{minipage}
			\begin{minipage}[b]{0.65in}
				\centerline{ \includegraphics[width=0.6in, height=.45in]{./Picture/Dataset/teapot/teapot(1).png} }
			\end{minipage}
			\begin{minipage}[b]{0.65in}
				\centerline{ \includegraphics[width=0.6in, height=.45in]{./Picture/Dataset/teapot/teapot(2).png} }
			\end{minipage}
			\begin{minipage}[b]{0.65in}
				\centerline{ \includegraphics[width=0.6in, height=.45in]{./Picture/Dataset/teapot/teapot(3).png} }
			\end{minipage}
			\begin{minipage}[b]{0.65in}
				\centerline{ \includegraphics[width=0.6in, height=.45in]{./Picture/Dataset/teapot/teapot(4).png} }
			\end{minipage}
			\begin{minipage}[b]{0.65in}
				\centerline{ \includegraphics[width=0.6in, height=.45in]{./Picture/Dataset/teapot/teapot(5).png} }
			\end{minipage}
			\begin{minipage}[b]{0.65in}
				\centerline{ \includegraphics[width=0.6in, height=.45in]{./Picture/Dataset/teapot/teapot(6).png} }
			\end{minipage}
			
		\end{figure}
	\end{exampleblock}
\end{frame}

\begin{frame}{Dataset}
	\begin{exampleblock}{Examples and groundtruth}
		\begin{figure}[htpb]
			\centering
			\subfigure[Cup]{
				\begin{minipage}[b]{1.4in}
					\includegraphics[width=1.2in]{./Picture/Example/cup-res.eps}
				\end{minipage}
			}
			\subfigure[Can]{
				\begin{minipage}[b]{1.4in}
					\includegraphics[width=1.2in]{./Picture/Example/can-res.eps}
				\end{minipage}
			}
			\subfigure[Teapot]{
				\begin{minipage}[b]{1.4in}
					\includegraphics[width=1.2in]{./Picture/Example/teapot-res.eps}
				\end{minipage}
			}
			\subfigure[Box]{
				\begin{minipage}[b]{1.4in}
					\includegraphics[width=1.2in]{./Picture/Example/box-res.eps}
				\end{minipage}
			}
			\subfigure[Bottle]{
				\begin{minipage}[b]{1.4in}
					\includegraphics[width=1.2in]{./Picture/Example/bottle-res.eps}
				\end{minipage}
			}
			\subfigure[Groundtruth]{
				\begin{minipage}[b]{1.4in}
					\includegraphics[width=1.2in]{./Picture/Example/groundtruth.eps}
				\end{minipage}
			}
			
			\label{fig:ResultShow} %% label for entire figure
		\end{figure}
	\end{exampleblock}
\end{frame}


\subsection{Evaluation Criteria}
\begin{frame}{Evaluation Criteria}
	\begin{exampleblock}{Jaccard Similarity}
		\centering{\LARGE{$J = \frac{|P \bigcap G|}{|P \bigcup G|}$}}
		%\begin{enumerate}
		\begin{itemize}
			\item \footnotesize{$P$ is predicted region, 
				$G$ is groundtruth; $|X|$ represents the area of rectangle $X$.}
		\end{itemize}
		%\end{enumerate}
	\end{exampleblock}
	\begin{exampleblock}{Compare with different registration methods}
		\centering{
			\includegraphics[width=\textwidth, height=1.7in]{./Picture/Example/table.png}
		}
	\end{exampleblock}
\end{frame}
%%%%%%%%%%%%%%%%%%%%%%%%%%%%%%%%%%%%%%%%%%%%%%%%%%%%%%%%%%%%%%%%%%%%%%%%%



%%%%%%%%%%%%%%%%%%%%%%%%%%%%%%%%%%%%%%%%%%%%%%%%%%%%%%%%%%%%%%%%%%%%%%%%%
\section{DEMO}
%%%%%%%%%%%%%%%%%%%%%%%%%%%%%%%%%%%%%%%%%%%%%%%%%%%%%%%%%%%%%%%%%%%%%%%%%
\subsection{}
\begin{frame}{Demo}
	\begin{exampleblock}{Video}
		%	\begin{figure}[ht]
		%	      	\includemovie[
		%	      	poster,
		%	      	text={\small(Loading demo.mp4)}
		%	      	]{4.68in}{2.7in}{./Video/demo.mp4}
		%	\end{figure}
	\end{exampleblock}
\end{frame}
%%%%%%%%%%%%%%%%%%%%%%%%%%%%%%%%%%%%%%%%%%%%%%%%%%%%%%%%%%%%%%%%%%%%%%%%%



%%%%%%%%%%%%%%%%%%%%%%%%%%%%%%%%%%%%%%%%%%%%%%%%%%%%%%%%%%%%%%%%%%%%%%%%%
\section{}
%%%%%%%%%%%%%%%%%%%%%%%%%%%%%%%%%%%%%%%%%%%%%%%%%%%%%%%%%%%%%%%%%%%%%%%%%
\begin{frame}[plain]
	\thispagestyle{empty}
	\begin{columns}
		\begin{column}{\paperwidth}
			\includegraphics[width=\paperwidth,height=\paperheight]{./Picture/Material/QA.jpg}
		\end{column}
	\end{columns}
\end{frame}

\begin{frame}[plain]
	\thispagestyle{empty}
	\begin{columns}
		\begin{column}{\paperwidth}
			\includegraphics[width=\paperwidth,height=\paperheight]{./Picture/Material/3Q.png}
		\end{column}
	\end{columns}
\end{frame}
%%%%%%%%%%%%%%%%%%%%%%%%%%%%%%%%%%%%%%%%%%%%%%%%%%%%%%%%%%%%%%%%%%%%%%%%%



\end{document}
